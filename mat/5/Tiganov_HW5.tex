\documentclass[12pt,a4paper]{article}
\usepackage[utf8]{inputenc}
\usepackage[russian]{babel}
\usepackage{amsmath,amssymb,amsthm}
\usepackage{geometry}
\usepackage{listings}
\usepackage{xcolor}
\geometry{left=2cm,right=2cm,top=2cm,bottom=2cm}

\title{Домашнее задание №5}
\date{}

\begin{document}

\maketitle
\Large\flushleft{Тиганов Вадим Игоревич, группа J3212\\ИСУ: 467701}

\lstset{
    breaklines=true,
    frame=single,
    numbers=left,
    numberstyle=\tiny,
    showstringspaces=false,
    backgroundcolor=\color{gray!10},
    language=Python,
}




\section*{Задача 1: Сжимающие отображения}

\subsection*{(a) $f(x) = \arctan(x)$}
Производная $f'(x) = \frac{1}{1 + x^2}$.
На $\mathbb{R}$ имеем $0 < f'(x) \leq 1$. Так как $\sup_{x \in \mathbb{R}} |f'(x)| = 1$ (достигается в $x=0$), отображение \textbf{не является сжимающим} на $\mathbb{R}$.
Оно является сжимающим на любом отрезке $[a, b]$, не содержащем $0$, так как на таком отрезке $\max |f'(x)| < 1$.

\subsection*{(b) $f(x) = \ln(1 + e^x)$}
Производная $f'(x) = \frac{e^x}{1 + e^x}$.
Для всех $x \in \mathbb{R}$ выполняется $0 < f'(x) < 1$. Однако $\sup_{x \in \mathbb{R}} f'(x) = \lim_{x \to \infty} \frac{e^x}{1 + e^x} = 1$.
Следовательно, отображение \textbf{не является сжимающим} на $\mathbb{R}$. Оно будет сжимающим на любом конечном отрезке $[a, b]$.

\subsection*{(c) $f(x) = \frac{x}{2} + \sin(x)$}
Производная $f'(x) = \frac{1}{2} + \cos(x)$.
Условие сжатия: $|\frac{1}{2} + \cos(x)| < 1$, что эквивалентно $-\frac{3}{2} < \cos(x) < \frac{1}{2}$.
Это выполняется, когда $x \in \bigcup_{k \in \mathbb{Z}} (\frac{\pi}{3} + 2\pi k, \frac{5\pi}{3} + 2\pi k)$.
Наибольшими отрезками, где отображение сжимающее, являются любые отрезки $[a,b]$, целиком лежащие внутри одного из этих интервалов.

\section*{Задача 2: Уравнение $x = e^{-x}$}

\paragraph{Доказательство единственности.}
Рассмотрим функцию $g(x) = x - e^{-x}$.
Её производная $g'(x) = 1 + e^{-x} > 1$ для всех $x \in \mathbb{R}$.
Так как $g'(x) > 0$, функция $g(x)$ строго монотонно возрастает.
Поскольку $\lim_{x \to -\infty} g(x) = -\infty$ и $\lim_{x \to +\infty} g(x) = +\infty$, по теореме о промежуточном значении существует корень. В силу строгой монотонности он единственен.

\paragraph{Итерационный процесс.}
Используем метод простых итераций: $x_{n+1} = e^{-x_n}$.
Начнём с $x_0 = 0$:
\begin{align*}
x_1 &= e^{-0} = 1 \\
x_2 &= e^{-1} \approx 0.3679 \\
x_3 &= e^{-0.3679} \approx 0.6922 \\
x_4 &= e^{-0.6922} \approx 0.5005 \\
x_5 &= e^{-0.5005} \approx 0.6062
\end{align*}
Последовательность сходится к решению $x \approx 0.56714$.\\
(Считал на Python:)
\begin{lstlisting}[language=Python]
import math

x_n = 0
sequence = [x_n]

for i in range(1, 10**6):
    x_n = math.exp(-x_n)
    sequence.append(round(x_n, 4))

print(sequence[0:6])
print(sequence[10**6 - 1])
# [0, 1.0, 0.3679, 0.6922, 0.5005, 0.6062]
# 0.5671
\end{lstlisting}

\section*{Задача 3: Производная обратной функции}
Пусть $f$ непрерывно дифференцируема в точке $a$, $f'(a) \neq 0$, и $g = f^{-1}$.
По определению производной:
\[ g'(y_0) = \lim_{y \to y_0} \frac{g(y) - g(y_0)}{y - y_0} \]
где $y_0 = f(a)$. Сделаем замену $x = g(y)$, $a = g(y_0)$. Из непрерывности $g$ следует, что $y \to y_0 \implies x \to a$.
\[ g'(y_0) = \lim_{x \to a} \frac{x - a}{f(x) - f(a)} = \lim_{x \to a} \frac{1}{\frac{f(x) - f(a)}{x - a}} \]
Так как предел знаменателя существует и равен $f'(a) \neq 0$:
\[ g'(y_0) = \frac{1}{\lim_{x \to a} \frac{f(x) - f(a)}{x - a}} = \frac{1}{f'(a)} \]





\end{document}