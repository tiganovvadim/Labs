\documentclass[12pt,a4paper]{article}
\usepackage[utf8]{inputenc}
\usepackage[russian]{babel}
\usepackage{amsmath,amssymb,amsthm}
\usepackage{geometry}
\usepackage{listings}
\usepackage{xcolor}
\geometry{left=2cm,right=2cm,top=2cm,bottom=2cm}

\title{Домашнее задание №6}
\date{}
\begin{document}

\maketitle
\Large\flushleft{Тиганов Вадим Игоревич, группа J3212\\ИСУ: 467701}

\section*{Задание 1}

Найти производную \( y'(x) \), если 
\[ e^y - e^x + x^2 + y^2 = 0 \]

\textbf{а) Через теорему об обратной функции:}

\( F(x, y) = e^y - e^x + x^2 + y^2 \)
\[
y'_x = -\frac{F'_x}{F'_y}
\]
\[
F'_x = -e^x + 2x,\quad F'_y = e^y + 2y
\]
\[
y'_x = -\frac{-e^x + 2x}{e^y + 2y} = \frac{e^x - 2x}{e^y + 2y}
\]

\textbf{б) Через дифференцирование равенства:}

Дифференцируем обе части уравнения по \( x \):
\[
e^y \cdot y'_x - e^x + 2x + 2y \cdot y'_x = 0
\]

Выразим \( y'_x \):
\[
y'_x (e^y + 2y) = e^x - 2x
\]
\[
y'_x = \frac{e^x - 2x}{e^y + 2y}
\]

\section*{Задание 2}

Дана система:
\[
\left\{
\begin{aligned}
&xu + yv - u^3 = 0\\\\
&x + y + u + v = 0
\end{aligned}
\right.
\]
Найти частные производные в точке \( A(1, 0, 1, -2) \):

\[
\frac{\partial u}{\partial x},\quad \frac{\partial u}{\partial y},\quad \frac{\partial v}{\partial x},\quad \frac{\partial v}{\partial y}
\]

Дифференцируем первое уравнение по \( x \):
\[
u + x \frac{\partial u}{\partial x} + y \frac{\partial v}{\partial x} - 3u^2 \frac{\partial u}{\partial x} = 0
\]

Дифференцируем первое уравнение по \( y \):
\[
x \frac{\partial u}{\partial y} + v + y \frac{\partial v}{\partial y} - 3u^2 \frac{\partial u}{\partial y} = 0
\]

Дифференцируем второе уравнение по \( x \):
\[
1 + \frac{\partial u}{\partial x} + \frac{\partial v}{\partial x} = 0
\]

Дифференцируем второе уравнение по \( y \):
\[
1 + \frac{\partial u}{\partial y} + \frac{\partial v}{\partial y} = 0
\]

Подставляем значения из точки \( A(1, 0, 1, -2) \):
\[
u = 1, v = -2, x = 1, y = 0
\]

Решаем систему уравнений:
\[
1 + \frac{\partial u}{\partial x} - 3 \frac{\partial u}{\partial x} = 0
\]
\[
\frac{\partial u}{\partial x} = \frac{1}{2}
\]

\[
1 + \frac{1}{2} + \frac{\partial v}{\partial x} = 0
\]
\[
\frac{\partial v}{\partial x} = -\frac{3}{2}
\]

\[
1 \cdot \frac{\partial u}{\partial y} - 2 - 3 \frac{\partial u}{\partial y} = 0
\]
\[
\frac{\partial u}{\partial y} = -1
\]

\[
1 - 1 + \frac{\partial v}{\partial y} = 0
\]
\[
\frac{\partial v}{\partial y} = 0
\]

\textbf{Ответ:}
\[
\frac{\partial u}{\partial x} = \frac{1}{2},\ 
\frac{\partial u}{\partial y} = -1,\ 
\frac{\partial v}{\partial x} = -\frac{3}{2},\ 
\frac{\partial v}{\partial y} = 0
\]

\end{document}
