\documentclass[12pt,a4paper]{article}
\usepackage[utf8]{inputenc}
\usepackage[T2A]{fontenc}
\usepackage[russian]{babel}
\usepackage{geometry}
\usepackage{fancyhdr}
\usepackage{setspace}
\usepackage{graphicx}
\usepackage{listings}
\usepackage{xcolor}
% \usepackage{hyperref} % удалено по инструкции
\usepackage{enumitem}
\usepackage{amsmath}
\usepackage{amsfonts}
\geometry{left=3cm, right=1.5cm, top=2cm, bottom=2cm}

\onehalfspacing

\usepackage{titlesec}
\titleformat{\section}{\centering\normalfont\bfseries}{\thesection}{1em}{}
\titleformat{\subsection}{\normalfont\bfseries}{\thesubsection}{1em}{}

\lstset{
    basicstyle=\small\ttfamily,
    breaklines=true,
    frame=single,
    numbers=left,
    numberstyle=\tiny,
    showstringspaces=false,
    backgroundcolor=\color{gray!10}
}

% \hypersetup{ % удалено по инструкции
%     colorlinks=true,
%     linkcolor=black,
%     urlcolor=blue,
%     citecolor=black
% }

\begin{document}

\begin{flushleft}
    Тиганов Вадим Игоревич \\
    ИСУ: 467701 \\
    Группа: J3212 \\ 
    Поток: 2.4
\end{flushleft}

$$
\textbf{1. Холопы после плена татар}
$$
\textbf{Задача:} В 1497 году группа холопов была захвачена в плен татарскими войсками во время набега на русские земли. Через несколько месяцев им удалось бежать из плена и вернуться к своим прежним господам. Господа требуют, чтобы холопы продолжили работать на них как прежде. Обязаны ли холопы, бежавшие из татарского плена, работать на прежнего господина?

\textbf{Ответ:} Согласно норме: «Холоп, попавший в плен к татарам и бежавший из плена, становится свободным» (п.56). Следовательно, правы: возвратившиеся из плена не обязаны работать на прежнего господина.

$$
\textbf{2. Переход Тимохи}
$$
\textbf{Задача:} Крестьянин Тимоха прожил у боярина Ивана Петровича ровно 2 года в лесной местности Московского уезда. Он построил дом, обработал землю и хочет перейти к другому боярину — Семёну Васильевичу. Боярин Иван Петрович требует уплаты «за двор». Какую сумму должен заплатить Тимоха «за двор» при переходе?

\textbf{Ответ:} Право перехода — в неделю до и неделю после осеннего Юрьева дня (26 ноября), с уплатой «за двор»; в лесной местности — полтина (п.57). За прожитые 2 года при уходе платится половина стоимости двора (там же). Итого:
$$
\text{Платёж} = \tfrac{1}{2}\cdot \text{стоимости двора} \quad (\text{лесная полоса обуславливает базовый размер «за двор», но для 2 лет — половина}).
$$

$$
\textbf{3. Потеря межевой отметки на сенокосе}
$$
\textbf{Задача:} Крестьянин Фёдор из деревни Заречье случайно потерял межевую отметку на общем сенокосе и не заметил, как его скот потравил сено соседа — крестьянина Григория. Григорий требует возмещения ущерба и штрафа. Фёдор утверждает, что это произошло случайно из-за потери межевой отметки. Как должен решаться этот спор о потраве сена?

\textbf{Ответ:} За межевые споры и потраву: убытки возмещает виновный; при нарушении межи — штраф и вознаграждение потерпевшему «по усмотрению» (пп.61–62). Компенсация возможна деньгами либо натурально (сеном), по оценке ущерба и усмотрению властей (волостель/посельский).

$$
\textbf{4. Суд поединком между боярами о разбое}
$$
\textbf{Задача:} Между двумя боярами — Андреем Михайловичем и Дмитрием Семёновичем — возник серьёзный спор. Андрей Михайлович обвиняет Дмитрия Семёновича в организации разбоя на дороге, где был ограблен его торговый караван. Дмитрий Семёнович отрицает обвинения. Оба боярина готовы доказать свою правоту. Могут ли они решить это дело судебным поединком?

\textbf{Ответ:} Дела о «поджоге, убийстве, разбое или воровстве» могут решаться судебным поединком; побеждённый уплачивает иск и несёт наказание/«продажу» (п.7, п.38). Ответчик, если престарел, увечен, духовного звания, женщина и т.п., может выставить наймита (п.49). Бояре могут выставить наймита при соответствующих условиях; иначе бьются лично.

$$
\textbf{5. Переход Семёна}
$$
\textbf{Задача:} Крестьянин Семён пришёл к боярину Никите Фёдоровичу за 3 месяца до Юрьева дня (26 ноября) и поселился в его деревне. Через месяц он решил, что условия жизни у этого боярина его не устраивают, и хочет уйти к другому господину. Может ли Семён покинуть боярина Никиту Фёдоровича через месяц после поселения?

\textbf{Ответ:} Переход допустим только в «Юрьев срок» (неделя до/после 26 ноября) с уплатой «за двор» (п.57). Приход за 3 месяца до срока не даёт права тут же уйти без уплаты; условия: соблюсти срок и оплату по прожитым годам (четвертями до полной стоимости к 4 годам).

$$
\textbf{6. Пересмотр дела купца Петрова}
$$
\textbf{Задача:} Купец Петров проиграл судебное дело против купца Сидорова на сумму 50 рублей по спору о поставке товаров. Петров считает решение несправедливым и хочет пересмотреть дело в вышестоящей инстанции. Какие дополнительные потери ожидают Петрова при пересмотре дела?

\textbf{Ответ:} За пересмотр дел с виновного взыскивается пошлина «2 гривны»; при сумме иска меньше рубля пошлин не берут; пошлина не берётся также с судных списков, холопов и по земельным делам (п.64). Иск — 50 рублей (>1 рубля). Итого потери Петрова:
$$
50 \text{ руб. (в пользу истца)}\;+\;2 \text{ гривны (пошлина за пересмотр)}.
$$

$$
\textbf{7. Отказ наместника выдать беглого холопа без доклада}
$$
\textbf{Задача:} К боярину Василию Ивановичу пришёл посол от другого князя с требованием выдать беглого холопа, который скрывается в его владениях. Наместник боярина отказался выдать холопа без доклада в вышестоящую инстанцию. Посол утверждает, что это нарушение межкняжеских соглашений. Правомерен ли отказ наместника выдать беглого холопа без доклада?

\textbf{Ответ:} Наместникам запрещено отпускать холопов, рабынь, а также выпускать из-под стражи «лихих людей» без утверждения вышестоящей инстанции (п.43). Вопрос о межкняжеской выдаче через доклад в вышестоящую инстанцию соответствует духу норм о компетенциях. Отказ до доклада — правомерен.

$$
\textbf{8. Неявка свидетеля и иск к приставу}
$$
\textbf{Задача:} Ремесленник-кузнец Иван был вызван в суд как свидетель по делу о краже. Пристав, который доставил ему повестку, указал неверную дату судебного заседания. Из-за этого Иван не явился в суд в назначенное время. Суд взыскал с него сумму иска и пошлины. Может ли Иван предъявить иск к приставу за неверное указание срока?

\textbf{Ответ:} Если свидетель не явится — с него взыскивается сумма иска, убытки и пошлины; но если неявка вызвана неверным указанием срока приставом, свидетель может предъявить иск к приставу (п.50). Следовательно, ремесленник вправе иск к приставу заявить.

$$
\textbf{9. Пошлины за правую грамоту на 3 холопов и 1 рабыню}
$$
\textbf{Задача:} Боярин Михаил Семёнович выиграл судебное дело и получает правую грамоту на освобождение 3 холопов и 1 рабыни от своих должников. Грамота выдаётся наместничьим судом. Какие пошлины должен заплатить боярин Михаил за получение этой правой грамоты?

\textbf{Ответ:} За правую/отпускную грамоту «с холопа и с рабы» у наместничьего/боярского суда: печатнику (или боярину, где применимо) — по 9 денег с человека; дьяку — по алтыну; подьячему — по 3 деньги (пп.17, 22–23, 40). На 4 человек:
$$
\text{Печатник: } 4 \times 9 = 36 \text{ денег},\quad
\text{Дьяк: } 4 \times 1 = 4 \text{ алтына},\quad
\text{Подьячий: } 4 \times 3 = 12 \text{ денег}.
$$

$$
\textbf{10. Иск о земле > 3 лет давности}
$$
\textbf{Задача:} Боярин Алексей Петрович подаёт иск против боярина Фёдора Никитича о спорной земле. Алексей утверждает, что эта земля принадлежала его предкам, но Фёдор владеет ею уже более 5 лет. Фёдор заявляет, что Алексей пропустил срок для подачи иска. Каковы шансы Алексея на успех в этом земельном споре?

\textbf{Ответ:} Исковая давность по земельным спорам боярин - боярин — 3 года (п.63). Поскольку владение оспаривается по фактам свыше 5 лет, шансы Алексея низкие; суд может отклонить иск по пропуску давности.

$$
\textbf{11. Уход Овсея 25 августа 1503 г.}
$$
\textbf{Задача:} Крестьянин Овсей прожил у боярина Степана Андреевича 3 года, но 25 августа 1503 года внезапно покинул его владения и перешёл к другому боярину. Боярин Степан Андреевич требует возвращения Овсея и возмещения ущерба. Какие права имеет боярин Степан в этой ситуации?

\textbf{Ответ:} Нарушен порядок «Юрьева срока»: переход разрешён только за неделю до/после 26 ноября (п.57). Боярин может требовать возвращения и взыскания «за двор» по нормам (четвертями/половиной/т.д. в зависимости от срока проживания) и применимых расходов.

$$
\textbf{12. Третий раз кража учеником кузнеца}
$$
\textbf{Задача:} Ученик кузнеца Антон, 16 лет, совершил третью кражу — украл у соседа медный котёл стоимостью 2 рубля. При задержании у него не оказалось никакого имущества для возмещения ущерба. Ранее он уже дважды был осуждён за кражи. Какое наказание ожидает Антона за третью кражу?

\textbf{Ответ:} Первый случай (кроме церковной/с убийством) — торговая казнь, «продажа» и возмещение; при отсутствии имущества — битьё кнутом и выдача в холопство до отработки (п.10). Вторичная кража — смертная казнь (п.11). Третья кража тем более влечёт смертную казнь; убытки возмещаются из имущества, которого у Антона нет — выдачи «до отработки» уже не будет, поскольку при вторичности — казнь.

$$
\textbf{13. Недельщик в Каширу: помощники и вознаграждение}
$$
\textbf{Задача:} Недельщик Иван отправляется в город Каширу с приставной грамотой для взыскания долга с местного купца. Расстояние до Каширы составляет 60 вёрст. Иван хочет взять с собой помощника из своей семьи. Может ли он это сделать и какое вознаграждение получит за поездку?

\textbf{Ответ:} Поездка с приставной — недельщик действует лично; разрешено посылать «лиц своей фамилии», а не наёмных (п.31). Пошлины «вне города» фиксированы по направлениям: до Каширы — полтина (п.30). Следовательно, он может взять из своей фамилии; вознаграждение: «езда» до Каширы — полтина, иное — по нормам приставных/коженого при необходимости (пп.28–30, 44).

$$
\textbf{14. Обвинение несколькими «добрыми людьми» в кражах}
$$
\textbf{Задача:} Несколько «добрых людей» (5-6 человек) из деревни обвиняют крестьянина Петра в совершении краж под присягой. Они утверждают, что видели, как Пётр крал имущество у соседей. Однако вещественных доказательств (поличного) нет, и Пётр отрицает все обвинения. Как должно решаться это дело?

\textbf{Ответ:} Обвинение 5–6 «добрыми людьми» с присягой при отсутствии доказательств прежних краж: если будет доказано, что ранее не крал, — обязанность удовлетворить иск без разбора (п.12). Если же пойман «с поличным» и присягой признан многократно кравшим — смертная казнь (п.13). По условию вещественных доказательств нет; при наличии присяги добрых людей и отрицании обвиняемого — взыскать сумму иска без дальнейшего разбора.

$$
\textbf{15. Подбивание против великого князя}
$$
\textbf{Задача:} Пять дворян из московского уезда — Иван, Пётр, Семён, Фёдор и Андрей — обвиняются в «крамоле». Их подозревают в том, что они подбивали других людей против великого князя Ивана III, распространяли слухи о его несправедливости и призывали к неповиновению. При доказанности обвинения какое наказание ожидает этих дворян?

\textbf{Ответ:} «Крамола» относится к числу тяжких: карается смертной казнью (п.9). При доказанности обвинения пяти дворян — смертная казнь.

$$
\textbf{16. Поединок о поджоге; убийство соперника}
$$
\textbf{Задача:} Между боярином Алексеем и купцом Михаилом возник спор о поджоге. Алексей обвиняет Михаила в том, что тот поджёг его амбар с зерном. Михаил отрицает обвинения. Суд назначил судебный поединок для решения спора. Во время поединка Алексей убил Михаила. Как должно решаться дело после убийства в судебном поединке?

\textbf{Ответ:} Дела о поджоге допускают разрешение судебным поединком; побеждённый несёт взыскание и наказание по усмотрению судьи/наместника (п.7, п.38). Убийство в рамках судебного поединка, если он законно назначен и проведён, — часть процесса: победитель выигрывает дело; с побеждённого взыскивается иск, пошлины, и он подлежит наказанию/«продаже».

$$
\textbf{17. Найм на месяц, 27 дней отработаны}
$$
\textbf{Задача:} Крестьянин Никита нанялся к боярину на месяц для строительства дома за плату 1 рубль. Он отработал 27 дней, но затем ушёл, не завершив работу. Боярин отказывается платить ему за отработанные дни, ссылаясь на то, что работа не завершена. Получит ли Никита вознаграждение за отработанные 27 дней?

\textbf{Ответ:} «Если наймит уйдёт до окончания обусловленной работы/срока, он лишается вознаграждения» (п.54). Поскольку не доработали месяц, суд подтвердит отказ в плате.

$$
\textbf{18. Вдова против должника; поединок}
$$
\textbf{Задача:} Вдова Мария, 35 лет, требует от купца Степана возврата долга в размере 10 рублей, который он взял у её покойного мужа год назад. Степан отказывается возвращать долг, утверждая, что уже вернул его покойному мужу. У Марии есть свидетели (послухи), которые подтверждают, что долг не был возвращён. Может ли вдова Мария участвовать в судебном поединке для решения этого спора?

\textbf{Ответ:} В делах по займам и личном оскорблении, если свидетель «показывает против ответчика», ответчик может либо биться с послухом, либо «под присягой» добровольно уплатить (п.48). Поскольку истец — вдова, она может нанять наймита; приносить присягу могут истец и послухи, а наймит обязан биться (п.52). Марии следует либо идти к поединку через наймита/послухов, либо добиться признания долга под присягой ответчика; при его упорстве — поединок.

$$
\textbf{19. Взятка наместнику; пересмотр у великого князя}
$$
\textbf{Задача:} Купец Илья проиграл дело в наместничьем суде и был оштрафован на 20 рублей. Позже выяснилось, что наместник получил от противной стороны взятку в размере 5 рублей за вынесение решения в их пользу. Дело пересматривается у великого князя. Каковы последствия для Ильи при пересмотре дела?

\textbf{Ответ:} Взятки судьям запрещены (пп.1, 67) ; при установлении ложности показаний/неправомерного разбора — решение признаётся недействительным, взысканное возвращается, дело рассматривается заново (п.19). Суд великого князя взыскивает пошлины «как у боярского суда» — по два алтына с рубля с виновного (п.21). Илья будет реабилитирован; штраф отменён; возможны взыскания за взятку и пересмотр.

$$
\textbf{20. Купец ограблен и не может вернуть долг}
$$
\textbf{Задача:} Купец Григорий взял у боярина Василия товары для торговли на сумму 50 рублей с условием возврата через полгода. По дороге в другой город Григорий был ограблен разбойниками, которые отняли у него все товары и деньги. Теперь он не может вернуть долг боярину Василию. Как должна решаться эта ситуация?

\textbf{Ответ:} Если товар/деньги для торговли погибнут «не по его вине» (потонет, сгорит, отнят войском), боярин велит выдать «грамоту об уплате суммы… в рассрочку и без процентов» (п.55). Документы: соответствующая грамота (правя/срочная по контексту; прямо указана «грамота об уплате… в рассрочку и без процентов»).

$$
\textbf{21. Доходы великого князя и его семьи от суда}
$$
\textbf{Задача:} Великий князь Иван III и его сыновья Василий и Юрий ведут активную судебную деятельность, рассматривая различные дела и выдавая различные грамоты. Из каких конкретных источников складываются их доходы от судебной деятельности? Какие пошлины они получают за разные виды судебных действий?

\textbf{Ответ:} При суде великого князя/его детей пошлины «в том же размере», что в боярском суде — по два алтына с рубля (п.21), а также за правые/бессудные/срочные/докладные грамоты: печатнику (9 денег с рубля), дьяку (алтын), подьячему (2–3 деньги) по видам (пп.22, 24–26). Следовательно, доходы состоят из судебных пошлин и пошлин за грамоты в указанных размерах.

\end{document}