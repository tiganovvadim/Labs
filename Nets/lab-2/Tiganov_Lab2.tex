\documentclass[12pt,a4paper]{article}
\usepackage[utf8]{inputenc}
\usepackage[T2A]{fontenc}
\usepackage[russian]{babel}
\usepackage{geometry}
\usepackage{fancyhdr}
\usepackage{setspace}
\usepackage{graphicx}
\usepackage{listings}
\usepackage{xcolor}
\usepackage{hyperref}
\usepackage{enumitem}
\usepackage{amsmath}
\usepackage{amsfonts}

\geometry{left=3cm, right=1.5cm, top=2cm, bottom=2cm}

\onehalfspacing

\usepackage{titlesec}
\titleformat{\section}{\centering\normalfont\bfseries}{\thesection}{1em}{}
\titleformat{\subsection}{\normalfont\bfseries}{\thesubsection}{1em}{}

\lstset{
    basicstyle=\small\ttfamily,
    breaklines=true,
    frame=single,
    numbers=left,
    numberstyle=\tiny,
    showstringspaces=false,
    backgroundcolor=\color{gray!10}
}

\hypersetup{
    colorlinks=true,
    linkcolor=black,
    urlcolor=blue,
    citecolor=black
}

\begin{document}

\begin{titlepage}
    \centering
    
    \textbf{Министерство науки и высшего образования Российской Федерации}\\
    \textbf{ФЕДЕРАЛЬНОЕ ГОСУДАРСТВЕННОЕ АВТОНОМНОЕ ОБРАЗОВАТЕЛЬНОЕ УЧРЕЖДЕНИЕ ВЫСШЕГО ОБРАЗОВАНИЯ}\\
    \textbf{НАЦИОНАЛЬНЫЙ ИССЛЕДОВАТЕЛЬСКИЙ УНИВЕРСИТЕТ ИТМО}
    
    \vspace{2cm}
    
    \textbf{Факультет ФТИИ}
    
    \vspace{1cm}
    
    \textbf{Дисциплина: «Сетевые технологии»}\\
    
    
    \vspace{2cm}
    
    {\Large \textbf{ПРАКТИЧЕСКАЯ РАБОТА № 2}}\\
    \vspace{0.5cm}
    {\large «Мониторинг сетевого трафика на хосте на примере работы с
    утилитами диагностики и мониторинга сетевых соединений в Linux»}
    
    \vspace{3cm}
    
    \begin{flushleft}
    \textbf{Выполнил:}\\
    Тиганов Вадим Игоревич, студент группы J3212\\
    ИСУ: 467701
{\hspace{5cm}}
    \end{flushleft}
    
    \vspace{1cm}
    
    \begin{flushleft}
    \textbf{Проверила:}\\
    Шиманская Галина Станиславовна\\
{\hspace{5cm}}
    \end{flushleft}
    
    \vfill
    
    Санкт-Петербург\\
    2025
    
\end{titlepage}

\tableofcontents
\newpage

\section{Цель работы}

Получить практические навыки мониторинга сетевого трафика и диагностирования сетевых соединений на хосте Linux с использованием консольных утилит: \texttt{ss}, \texttt{lsof}, \texttt{tcpdump}/\texttt{tshark}, \texttt{iftop}/\texttt{nload}/\texttt{bmon}, а также базовых средств \texttt{ip} и \texttt{ethtool}.



\section{Артефакты выполнения}

% Ниже оставлены структурные места для вставки артефактов. Заполните скриншотами/листингами.

\subsection{Часть 1. Настройка инфраструктуры}
ВМ c7-1/c7-2, NAT в VirtualBox, DHCP-адреса, проверка ping внешней сети, установка утилит (bmon/nload/iftop, nethogs, mtr, traceroute, vnstat, nc)

\subsection{Часть 2. Диагностика соединения}
Команды и выводы по \texttt{ping}, запуск \texttt{mtr} к \texttt{www.itmo.ru}, сохранение расширенной статистики на 40 пакетов:
\begin{figure}[h]
    \centering
    \includegraphics[width=1\textwidth]{9.jpg}
    \caption{Вывод команды \texttt{ping}}
\end{figure}

\subsection{Часть 3. Работа с Wireshark}
Настройка захвата (ограничение 5 МБ), статистика — самый активный узел, широковещательный «говорун», самый активный TCP-порт; графики Io Graphs (TCP/UDP вместе); Flow Graph по HTTPS; фильтры отображения для DNS/кадров хоста/широковещаний:

\begin{center}
    \includegraphics[width=1\textwidth]{11.jpg}
    wireshark 1
\end{center}

\begin{center}
    \includegraphics[width=1\textwidth]{12.jpg}
    wireshark 2
\end{center}

\begin{center}
    \includegraphics[width=1\textwidth]{14.jpg}
    wireshark 3
\end{center}

\begin{center}
    \includegraphics[width=1\textwidth]{15.jpg}
    wireshark 4 
\end{center}


\subsection{Часть 4. Определение маршрута прохождения пакета}
\textit{Место для: команды \texttt{traceroute} с ICMP/UDP/TCP до 8.8.8.8 и проверка фрагментации IPv4 — команды и выводы}\vspace{0.5cm}
\begin{center}
    \includegraphics[width=1\textwidth]{10.jpg}
    traceroute 8.8.8.8
\end{center}

\subsection{Часть 5. Текущий мониторинг сетевых интерфейсов}
Место для: \texttt{ping -f} с c7-2 на внутренний интерфейс c7-1;
\begin{center}
    \includegraphics[width=1\textwidth]{1.jpg}
    Проверим ip адреса
\end{center}
\begin{center}
    \includegraphics[width=1\textwidth]{2.jpg}
    Пингуем для проверки соединения
\end{center}
\subsection{Часть 6. Диагностика работы приложений через сеть}
SSH-сессии; вывод \texttt{netstat}/\texttt{ss} — прослушиваемые порты и установленные соединения; скрипт агрегации соединений по порту (по умолчанию 22); закрытие сессий — вывод:
\begin{center}
    \includegraphics[width=1\textwidth]{3.jpg}
    tcpdump
\end{center}
\begin{center}
    \includegraphics[width=1\textwidth]{4.jpg}
    netstat
\end{center}
\begin{center}
    \includegraphics[width=1\textwidth]{5.jpg}
    bmon
\end{center}
\begin{center}
    \includegraphics[width=1\textwidth]{6.jpg}
    Дополнительный мониторинг
\end{center}
\begin{center}
    \includegraphics[width=1\textwidth]{8.jpg}
    Отправим Hello World по SSH
\end{center}

\section{Ответы на вопросы и задания}

\begin{enumerate}
    \item \textbf{По какому протоколу работает утилита mtr? Как это можно определить?}\\
    По умолчанию \texttt{mtr} в Linux использует \textbf{ICMP Echo} (Echo Request/Reply) аналогично \texttt{traceroute -I}. Определить можно по захвату \texttt{tcpdump/wireshark} (видны ICMP Echo), либо явно переключая режимы \texttt{-u} (UDP) и \texttt{-T} (TCP) и наблюдая изменение типа пакетов.

    \item \textbf{Опишите значения столбцов статистики, выводимой утилитой mtr. Какие еще статистики доступны в mtr кроме основных?}\\
    Базовые столбцы: \textit{Loss\%} — потери по хопу; \textit{Snt} — число зондов; \textit{Last} — время последнего ответа; \textit{Avg} — средняя RTT; \textit{Best} — минимальная RTT; \textit{Wrst} — максимальная RTT; \textit{StDev} — стандартное отклонение. Дополнительно доступны отчеты (\texttt{--report}, \texttt{--report-cycles}), форматы \texttt{--json}/\texttt{--xml}, отображение \texttt{--show-ips}/AS/GeoIP, а также \textit{Jitter} в некоторых сборках.

    \item \textbf{Какие типы кадров Ethernet бывают, в чем их отличия?}\\
    \textit{Ethernet II} (поле EtherType, наиболее распространен); \textit{IEEE 802.3 LLC} (длина + LLC заголовок); \textit{802.3 SNAP} (LLC+SNAP для индикации протокола); \textit{802.1Q VLAN}/QinQ (теги VLAN); служебные кадры типа \textit{PAUSE 802.3x}, \textit{LLDP}. Отличаются форматом полей заголовка и наличием тегов/LLC.

    \item \textbf{На какие адреса сетевого уровня осуществляются широковещательные рассылки?}\\
    В IPv4: \textbf{255.255.255.255} (ограниченный широковещательный) и \textbf{направленный широковещательный} адрес сети (например, 192.168.1.255/24). В IPv6 широковещания нет — используется многоадресная рассылка (\texttt{ff00::/8}).

    \item \textbf{На какой канальный адрес осуществляются широковещательные рассылки?}\\
    На MAC-адрес \textbf{ff:ff:ff:ff:ff:ff}.

    \item \textbf{Для чего применяются перехваченные широковещательные рассылки в Части 3?}\\
    Примеры: \textit{ARP Request} (разрешение IP→MAC), \textit{DHCP Discover/Offer} (получение параметров IP-сети), \textit{mDNS/LLMNR/NBNS} (локальное разрешение имён), а также сервисные рассылки (например, \textit{STP/LLDP}). Требовалось определить назначение минимум трёх таких рассылок.

    \item \textbf{Как изменяется загрузка интерфейса в Части 5. п. 3? Почему?}\\
    При увеличении размера ICMP-пакета (100→60100 байт) \textbf{доля заголовочного оверхеда уменьшается}, а \textbf{битовая загрузка} интерфейса при flood растёт до упора в пропускную способность/ограничения ядра; частота пакетов (pps) падает. Итог: использование канала увеличивается примерно пропорционально полезной нагрузке до достижения пределов CPU/линейной скорости.

    \item \textbf{На каком уровне модели OSI работает vnstat?}\\
    \texttt{vnstat} собирает счётчики интерфейсов из ядра (\texttt{/proc/net/dev}), то есть оперирует \textbf{на канальном уровне (L2)} независимо от протоколов L3/L4.
\end{enumerate}

\section{Использование GAI}

Был ли использован в ходе выполнения практической работы GAI (ChatGPT, YandexGPT и др.)?

Да, для помощи в структуре отчета, формулировках теоретических ответов и оформления LaTeX.

\textbf{Цели использования:}

\begin{itemize}
    \item Подготовка шаблона разделов и заглушек под артефакты
    \item Вычитка и сжатие теоретических формулировок
\end{itemize}


\textbf{Оценка качества ответов моделей:}

Качество ответов ChatGPT было высоким. Модель:
\begin{itemize}
    \item Помогла структурировать материал и сделать листинг более читаемым
    \item Дала корректные примеры фильтров и ключей для утилит (особенно для wireshark)
\end{itemize}

\section{Рефлексия}

Что вы узнали нового из работы? Как, по-вашему, эти знания или навыки могут пригодиться в будущей профессиональной деятельности?

В ходе выполнения данной лабораторной работы я приобрел следующие знания и навыки:

\textbf{Новые знания:}
\begin{itemize}
    \item Различия между инвентаризацией сокетов (\texttt{ss}/\texttt{lsof}) и пакетным анализом \\ (\texttt{tcpdump}/\texttt{tshark})
    \item Интерпретация состояний TCP и счетчиков интерфейса
    \item Подходы к фильтрации трафика с помощью BPF-выражений
\end{itemize}

\textbf{Практические навыки:}
\begin{itemize}
    \item Быстрая диагностика «что слушает порт» и «кто держит соединение»
    \item Прицельный захват и разбор трафика, запись/чтение PCAP
    \item Анализ пропускной способности и выявление узких мест по интерфейсам
\end{itemize}

\textbf{Применение в профессиональной деятельности:}
\begin{itemize}
    \item \textbf{Системное администрирование}: оперативная диагностика сетевых инцидентов
    \item \textbf{DevOps/SRE}: воспроизведение проблем, профилирование сетевых зависимостей сервисов
    \item \textbf{Безопасность}: сетевой Threat Hunting и анализ аномалий
\end{itemize}

\section{Список использованных источников}

\begin{enumerate}
    \item man-pages: \texttt{ss(8)}, \texttt{tcpdump(8)}, \texttt{tshark(1)}, \texttt{lsof(8)}, \texttt{ip-link(8)}, \texttt{ethtool(8)}
    \item \url{https://www.tcpdump.org/manpages/tcpdump.1.html}
    \item \url{https://www.wireshark.org/docs/wsug_html_chunked/}
    \item \url{https://www.kernel.org/doc/Documentation/networking/}
\end{enumerate}

\end{document}