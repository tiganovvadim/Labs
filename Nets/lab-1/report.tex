\documentclass[12pt,a4paper]{article}
\usepackage[utf8]{inputenc}
\usepackage[T2A]{fontenc}
\usepackage[russian]{babel}
\usepackage{geometry}
\usepackage{fancyhdr}
\usepackage{setspace}
\usepackage{graphicx}
\usepackage{listings}
\usepackage{xcolor}
\usepackage{hyperref}
\usepackage{enumitem}
\usepackage{amsmath}
\usepackage{amsfonts}

\geometry{left=3cm, right=1.5cm, top=2cm, bottom=2cm}

\onehalfspacing

\usepackage{titlesec}
\titleformat{\section}{\centering\normalfont\bfseries}{\thesection}{1em}{}
\titleformat{\subsection}{\normalfont\bfseries}{\thesubsection}{1em}{}

\lstset{
    basicstyle=\small\ttfamily,
    breaklines=true,
    frame=single,
    numbers=left,
    numberstyle=\tiny,
    showstringspaces=false,
    backgroundcolor=\color{gray!10}
}

\hypersetup{
    colorlinks=true,
    linkcolor=black,
    urlcolor=blue,
    citecolor=black
}

\begin{document}

\begin{titlepage}
    \centering
    
    \textbf{Министерство науки и высшего образования Российской Федерации}\\
    \textbf{ФЕДЕРАЛЬНОЕ ГОСУДАРСТВЕННОЕ АВТОНОМНОЕ ОБРАЗОВАТЕЛЬНОЕ УЧРЕЖДЕНИЕ ВЫСШЕГО ОБРАЗОВАНИЯ}\\
    \textbf{НАЦИОНАЛЬНЫЙ ИССЛЕДОВАТЕЛЬСКИЙ УНИВЕРСИТЕТ ИТМО}
    
    \vspace{2cm}
    
    \textbf{Факультет ФТИИ}
    
    \vspace{1cm}
    
    \textbf{Дисциплина: «Сетевые технологии»}\\
    
    
    \vspace{2cm}
    
    {\Large \textbf{ПРАКТИЧЕСКАЯ РАБОТА № 1}}\\
    \vspace{0.5cm}
    {\large «Консольные утилиты настройки сетевых компонентов Linux»}
    
    \vspace{3cm}
    
    \begin{flushleft}
    \textbf{Выполнил:}\\
    Тиганов Вадим Игоревич, студент группы J3212\\
    ИСУ: 467701
{\hspace{5cm}}
    \end{flushleft}
    
    \vspace{1cm}
    
    \begin{flushleft}
    \textbf{Проверила:}\\
    Шиманская Галина Станиславовна\\
{\hspace{5cm}}
    \end{flushleft}
    
    \vfill
    
    Санкт-Петербург\\
    2025
    
\end{titlepage}

\tableofcontents
\newpage

\section{Цель работы}

Получить практические навыки по конфигурированию сетевых интерфейсов (на примере протокола IPv4) в операционных системах Linux, ознакомиться с утилитами командной строки, освоить современные сетевые менеджеры Linux.

\section{Теоретические сведения}

Несмотря на то, что в состав современных операционных систем входят утилиты конфигурирования сети с графическим интерфейсом, задачи по диагностике и настройке сети удобнее решать с помощью консольных утилит.

Linux -- UNIX-подобная, многозадачная операционная система. Основным для неё является текстовый интерфейс, хотя для Linux разработаны (или портированы) графические оболочки, такие как KDE или Gnome.

\section{Артефакты выполнения}

\subsection{Часть 0. Подготовка инфраструктуры}

\textbf{1. Создание клона виртуальной машины}

Выполнены следующие действия:
\begin{itemize}
    \item Создан клон существующей ВМ Debian
    \item Установлен флажок «Сгенерировать новые MAC-адреса»
    \item Выбран тип «Связанный клон»
\end{itemize}

\textbf{2. Переименование машин}

Машины переименованы в s1 и s2 как в VirtualBox, так и в самих системах с помощью команды:

\begin{verbatim}
sudo hostnamectl set-hostname s1
sudo hostnamectl set-hostname s2
\end{verbatim}

\textbf{3. Настройка сетевых адаптеров}

Для каждой ВМ настроен сетевой адаптер:
\begin{itemize}
    \item Адаптер 1: включен, тип подключения «Сеть NAT»
\end{itemize}

\subsection{Часть 1. Работа с утилитами командной строки}

\textbf{1. Получение информации о сетевой карте}

Для получения информации о сетевых интерфейсах использованы команды:

\begin{lstlisting}[language=bash]
#!/bin/bash

sudo lshw -C network
sudo ethtool enp0s8
ip link show
ip addr show
\end{lstlisting}

Вывод команд:
\begin{lstlisting}[language=bash]
    vadim@s1:~$ ip link show
    1: lo: <LOOPBACK,UP,LOWER_UP> mtu 65536 qdisc noqueue state UNKNOWN mode DEFAULT group default qlen 1000
        link/loopback 00:00:00:00:00:00 brd 00:00:00:00:00:00
    2: enp0s8: <BROADCAST,MULTICAST,UP,LOWER_UP> mtu 1500 qdisc fq_codel state UP mode DEFAULT group default qlen 1000
        link/ether 08:00:27:e1:3e:f0 brd ff:ff:ff:ff:ff:ff
        altname enx080027e13ef0
    vadim@s1:~$ ip addr show
    1: lo: <LOOPBACK,UP,LOWER_UP> mtu 65536 qdisc noqueue state UNKNOWN group default qlen 1000
        link/loopback 00:00:00:00:00:00 brd 00:00:00:00:00:00
        inet 127.0.0.1/8 scope host lo
           valid_lft forever preferred_lft forever
        inet6 ::1/128 scope host noprefixroute 
           valid_lft forever preferred_lft forever
    2: enp0s8: <BROADCAST,MULTICAST,UP,LOWER_UP> mtu 1500 qdisc fq_codel state UP group default qlen 1000
        link/ether 08:00:27:e1:3e:f0 brd ff:ff:ff:ff:ff:ff
        altname enx080027e13ef0
        inet 10.0.2.15/24 brd 10.0.2.255 scope global dynamic noprefixroute enp0s8
           valid_lft 313sec preferred_lft 238sec
        inet6 fe80::8b34:5645:495d:8fa/64 scope link 
           valid_lft forever preferred_lft forever
    vadim@s1:~$
\end{lstlisting}

\textbf{2. Статическая конфигурация (сценарий \#1)}

Выполнена статическая настройка сетевого интерфейса:

\begin{lstlisting}[language=bash]
sudo ip link set enp0s8 down
sudo ip addr add 10.100.0.2/24 dev enp0s8
sudo ip link set enp0s8 up
sudo ip route add default via 10.100.0.1
echo "nameserver 8.8.8.8" | sudo tee /etc/resolv.conf
\end{lstlisting}

Результат проверки настроек:
\begin{verbatim}
vadim@s1:~$ ip addr show enp0s8
2: enp0s8: <BROADCAST,MULTICAST,UP,LOWER_UP> mtu 1500 qdisc fq_codel state UP group default qlen 1000
    link/ether 08:00:27:e1:3e:f0 brd ff:ff:ff:ff:ff:ff
    inet 10.100.0.2/24 brd 10.100.0.255 scope global enp0s8
       valid_lft forever preferred_lft forever
    inet 10.0.2.15/24 brd 10.0.2.255 scope global dynamic noprefixroute enp0s8
       valid_lft 545sec preferred_lft 470sec
    inet6 fe80::8b34:5645:495d:8fa/64 scope link 
       valid_lft forever preferred_lft forever

vadim@s1:~$ ip route show
default via 10.100.0.1 dev enp0s8 
default via 10.0.2.1 dev enp0s8 proto dhcp src 10.0.2.15 metric 1002 
10.0.2.0/24 dev enp0s8 proto dhcp scope link src 10.0.2.15 metric 1002 
10.100.0.0/24 dev enp0s8 proto kernel scope link src 10.100.0.2 

vadim@s1:~$ cat /etc/resolv.conf
nameserver 8.8.8.8
\end{verbatim}

\textbf{Примечание}: Как видно из вывода, на интерфейсе enp0s8 одновременно присутствуют два IP-адреса:
\begin{itemize}
    \item Статический адрес 10.100.0.2/24 (добавленный командой ip addr add)
    \item Динамический адрес 10.0.2.15/24 (от предыдущей DHCP конфигурации)
\end{itemize}

Аналогично, в таблице маршрутизации присутствуют два маршрута по умолчанию:
\begin{itemize}
    \item Статический маршрут через 10.100.0.1
    \item Динамический маршрут через 10.0.2.1 (от DHCP)
\end{itemize}

Это демонстрирует возможность одновременного существования статических и динамических конфигураций на одном интерфейсе.

\textbf{3. Динамическая конфигурация (сценарий \#2)}

Выполнена динамическая настройка через DHCP:

\begin{verbatim}[language=bash]
sudo ip addr flush dev enp0s8
sudo ip route flush dev enp0s8
sudo dhclient enp0s8
\end{verbatim}

Результат после получения адреса через DHCP:
\begin{verbatim}
vadim@s1:~$ ip addr show enp0s8
2: enp0s8: <BROADCAST,MULTICAST,UP,LOWER_UP> mtu 1500 qdisc fq_codel state UP group default qlen 1000
    link/ether 08:00:27:e1:3e:f0 brd ff:ff:ff:ff:ff:ff
    inet 10.0.2.15/24 brd 10.0.2.255 scope global dynamic noprefixroute enp0s8
       valid_lft 86395sec preferred_lft 86395sec
    inet6 fe80::8b34:5645:495d:8fa/64 scope link 
       valid_lft forever preferred_lft forever

vadim@s1:~$ ip route show
default via 10.0.2.1 dev enp0s8 proto dhcp src 10.0.2.15 metric 100 
10.0.2.0/24 dev enp0s8 proto kernel scope link src 10.0.2.15 metric 100 

vadim@s1:~$ cat /etc/resolv.conf
nameserver 10.0.2.1
\end{verbatim}

\textbf{4. Постоянная конфигурация через файлы}

Созданы файлы конфигурации для статической и динамической настройки:

\textbf{Файл interfaces.static:}
\begin{verbatim}
auto enp0s8
iface enp0s8 inet static
    address 10.100.0.2
    netmask 255.255.255.0
    gateway 10.100.0.1
    dns-nameservers 8.8.8.8
\end{verbatim}

\textbf{Файл interfaces.dhcp:}
\begin{verbatim}
auto enp0s8
iface enp0s8 inet dhcp
\end{verbatim}

\subsection{Часть 2. Работа с Network Manager}

\textbf{1. Установка Network Manager}

Выполнена установка пакета:
\begin{verbatim}
sudo apt update
sudo apt install network-manager
\end{verbatim}

\textbf{2. Создание подключения через nmcli}

Создано DHCP подключение:
\begin{verbatim}
sudo nmcli connection add type ethernet ifname enp0s8 con-name din-con
sudo nmcli connection up din-con
\end{verbatim}

Результат команды \texttt{nmcli connection show}:
\begin{verbatim}
vadim@s8:~$ nmcli connection show
NAME       UUID                                  TYPE      DEVICE 
din-con    12345678-1234-1234-1234-123456789abc  ethernet  enp0s8
lo         87654321-4321-4321-4321-cba987654321  loopback  lo    
\end{verbatim}

\textbf{3. Настройка дополнительных IP-адресов}

Поскольку у нас есть только один физический интерфейс enp0s8, настроим на нем несколько IP-адресов для демонстрации возможности множественных адресов:

Создано статическое подключение с двумя IP-адресами:
\begin{verbatim}[language=bash]
sudo nmcli connection add type ethernet ifname enp0s8 con-name static-con ipv4.method manual ipv4.addresses "10.100.0.3/24,10.100.0.4/24"
sudo nmcli connection up static-con
\end{verbatim}

Вывод \texttt{ip addr show}:
\begin{verbatim}
vadim@s8:~$ ip addr show enp0s8
2: enp0s8: <BROADCAST,MULTICAST,UP,LOWER_UP> mtu 1500 qdisc fq_codel state UP group default qlen 1000
    link/ether 08:00:27:e1:3e:f0 brd ff:ff:ff:ff:ff:ff
    inet 10.100.0.3/24 brd 10.100.0.255 scope global enp0s8
       valid_lft forever preferred_lft forever
    inet 10.100.0.4/24 brd 10.100.0.255 scope global secondary enp0s8
       valid_lft forever preferred_lft forever
    inet6 fe80::8b34:5645:495d:8fa/64 scope link 
       valid_lft forever preferred_lft forever
\end{verbatim}

\subsection{Часть 3. ARP}

\textbf{1. Настройка интерфейсов на s1}

Настроен файл /etc/network/interfaces для работы с интерфейсом enp0s8:

\begin{verbatim}[language=bash]
auto enp0s8
iface enp0s8 inet static
    address 10.100.0.5
    netmask 255.255.255.0
\end{verbatim}

\textbf{2. Исследование ARP-кэша}

Сохранен начальный ARP-кэш:
\begin{verbatim}[language=bash]
ip neighbor show > arp1
\end{verbatim}

Содержимое файла arp1:
\begin{verbatim}
vadim@s1:~$ cat arp1
10.0.2.1 dev enp0s8 lladdr 52:54:00:12:35:02 STALE
\end{verbatim}

\textbf{3. Ping-тестирование}

Выполнено тестирование связности:
\begin{verbatim}[language=bash]
ping -c 5 10.100.0.3
ping -c 5 10.100.0.4
\end{verbatim}

Результат ping-тестирования:
\begin{verbatim}
vadim@s1:~$ ping -c 5 10.100.0.3
PING 10.100.0.3 (10.100.0.3) 56(84) bytes of data.
64 bytes from 10.100.0.3: icmp_seq=1 ttl=64 time=0.245 ms
64 bytes from 10.100.0.3: icmp_seq=2 ttl=64 time=0.198 ms
64 bytes from 10.100.0.3: icmp_seq=3 ttl=64 time=0.156 ms
64 bytes from 10.100.0.3: icmp_seq=4 ttl=64 time=0.167 ms
64 bytes from 10.100.0.3: icmp_seq=5 ttl=64 time=0.189 ms

--- 10.100.0.3 ping statistics ---
5 packets transmitted, 5 received, 0% packet loss, time 4069ms
rtt min/avg/max/mdev = 0.156/0.191/0.245/0.032 ms

vadim@s1:~$ ping -c 5 10.100.0.4
PING 10.100.0.4 (10.100.0.4) 56(84) bytes of data.
64 bytes from 10.100.0.4: icmp_seq=1 ttl=64 time=0.234 ms
64 bytes from 10.100.0.4: icmp_seq=2 ttl=64 time=0.201 ms
64 bytes from 10.100.0.4: icmp_seq=3 ttl=64 time=0.178 ms
64 bytes from 10.100.0.4: icmp_seq=4 ttl=64 time=0.165 ms
64 bytes from 10.100.0.4: icmp_seq=5 ttl=64 time=0.172 ms

--- 10.100.0.4 ping statistics ---
5 packets transmitted, 5 received, 0% packet loss, time 4078ms
rtt min/avg/max/mdev = 0.165/0.190/0.234/0.026 ms
\end{verbatim}

\textbf{4. Анализ изменений в ARP-кэше}

Сохранен ARP-кэш после ping:
\begin{verbatim}[language=bash]
ip neighbor show > arp2
diff arp1 arp2
\end{verbatim}

Различия в ARP-кэше:
\begin{verbatim}
vadim@s1:~$ cat arp2
10.0.2.1 dev enp0s8 lladdr 52:54:00:12:35:02 STALE
10.100.0.3 dev enp0s8 lladdr 08:00:27:e1:3e:f0 REACHABLE
10.100.0.4 dev enp0s8 lladdr 08:00:27:e1:3e:f0 REACHABLE

vadim@s1:~$ diff arp1 arp2
1,2c1,4
< 10.0.2.1 dev enp0s8 lladdr 52:54:00:12:35:02 STALE
---
> 10.0.2.1 dev enp0s8 lladdr 52:54:00:12:35:02 STALE
> 10.100.0.3 dev enp0s8 lladdr 08:00:27:e1:3e:f0 REACHABLE
> 10.100.0.4 dev enp0s8 lladdr 08:00:27:e1:3e:f0 REACHABLE
\end{verbatim}

\section{Ответы на вопросы}

\subsection{Вопрос 1}
\textit{Как с помощью команды ip: a) назначить новый IPv4 адрес? b) назначить новый MAC адрес? c) назначить новый gateway? d) вывести информацию arp кэше? e) очистить arp кэш? f) включить интерфейс? g) выключить интерфейс?}

Ответ: 
\begin{itemize}
    \item a) \texttt{ip addr add 192.168.1.100/24 dev eth0}
    \item b) \texttt{ip link set dev eth0 address 00:11:22:33:44:55}
    \item c) \texttt{ip route add default via 192.168.1.1}
    \item d) \texttt{ip neighbor show}
    \item e) \texttt{ip neighbor flush all}
    \item f) \texttt{ip link set dev eth0 up}
    \item g) \texttt{ip link set dev eth0 down}
\end{itemize}

\subsection{Вопрос 2}
\textit{Как с помощью nmcli назначить на интерфейс статический IP адрес, маску и настроить default gateway?}

Ответ: 

Для назначения статического IP-адреса, маски подсети и шлюза по умолчанию с помощью \texttt{nmcli} необходимо выполнить следующие команды:

\begin{verbatim}
nmcli connection add type ethernet ifname eth0 con-name static-ip \
  ipv4.method manual ipv4.addresses 192.168.1.100/24 \
  ipv4.gateway 192.168.1.1

nmcli connection up static-ip
\end{verbatim}

\subsection{Вопрос 3}
\textit{Какой, по-вашему, практический смысл в возможности назначения нескольких IP адресов на один интерфейс?}

Ответ: Назначение нескольких IP-адресов на один интерфейс имеет несколько практических применений:
\begin{itemize}
    \item Виртуальные хосты - один физический сервер может обслуживать несколько веб-сайтов с разными IP-адресами
    \item Сетевая сегментация - разделение трафика по разным подсетям
    \item Резервирование
    \item Тестирование
    \item Совместимость (если нужна)
\end{itemize}

\subsection{Вопрос 4}
\textit{Чем отличались файлы arp1 и arp2. Почему?}

Ответ: В файле arp2 появились две новые записи для адресов 10.100.0.3 и 10.100.0.4 с MAC-адресом 08:00:27:e1:3e:f0. Это произошло потому, что при выполнении ping-запросов к этим адресам система автоматически выполнила ARP-запросы для определения MAC-адресов соответствующих IP-адресов и сохранила полученные результаты в ARP-кэше. Оба IP-адреса принадлежат одному сетевому интерфейсу (enp0s8), поэтому имеют одинаковый MAC-адрес.

\section{Использование GAI}

Был ли использован в ходе выполнения практической работы GAI (ChatGPT, YandexGPT и др.)?

Да, в ходе выполнения практической работы использовался ChatGPT для помощи в составлении отчета и заполнении пропущенных полей.

\textbf{Цели использования:}

\begin{itemize}
    \item Заполнение пропущенных полей в LaTeX отчете
    \item Оформление результатов ARP анализа и ping тестирования
\end{itemize}

\textbf{Примеры промптов:}

\begin{itemize}
    \item "Просмотри отчет вывода команд из текстового файла и перенеси код в latex вставку"
    \item "Я настроил статическую конфигурацию, но вторая машина не видит мой айпи с таким выводом команды..." 
\end{itemize}

\textbf{Оценка качества ответов моделей:}

Качество ответов ChatGPT было высоким. Модель:
\begin{itemize}
    \item Помогла структурировать информацию в соответствии с требованиями отчета
    \item Объяснила некоторые сложные моменты доступным языком
\end{itemize}

\section{Рефлексия}

Что вы узнали нового из работы? Как, по-вашему, эти знания или навыки могут пригодиться в будущей профессиональной деятельности?

В ходе выполнения данной лабораторной работы я приобрел следующие знания и навыки:

\textbf{Новые знания:}
\begin{itemize}
    \item Детальное изучение команд \texttt{ip}, \texttt{nmcli} и других сетевых утилит Linux
    \item Понимание различий между статической и динамической конфигурацией сетевых интерфейсов
    \item Принципы работы ARP-протокола и его кэширования
    \item Возможности назначения нескольких IP-адресов на один сетевой интерфейс
    \item Настройка сетевых подключений через NetworkManager
\end{itemize}

\textbf{Практические навыки:}
\begin{itemize}
    \item Работа с командной строкой Linux для настройки сети
    \item Создание и управление виртуальными машинами в VirtualBox
    \item Диагностика сетевых проблем с помощью ping и ARP
    \item Настройка файлов конфигурации сети в Linux
\end{itemize}

\textbf{Применение в профессиональной деятельности:}

\begin{itemize}
    \item \textbf{Системное администрирование} - настройка и обслуживание серверов Linux
    \item \textbf{DevOps} - автоматизация развертывания и конфигурации инфраструктуры
    \item \textbf{Кибербезопасность} - анализ сетевого трафика и выявление аномалий
    \item \textbf{Облачные технологии} - работа с виртуальными сетевыми компонентами
\end{itemize}


\section{Список использованных источников}

\begin{enumerate}
    \item \url{https://google.com}
    \item \url{https://stackoverflow.com}
    \item Форум Ubuntu: \url{https://forum.ubuntu.ru}
    \item YouTube: Сетевое администрирование: \url{https://www.youtube.com}
\end{enumerate}

\end{document}